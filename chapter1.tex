% !TEX root = thesis.tex

\chapter{Background and related work}
\label{cha:background}
\todo{Purpose of this study? What do we want to achieve? What should we do next?}


\section{Wikipedia}
\label{sec:wiki}
Wikipedia is a free-access, free-content Internet encyclopedia, supported and hosted by the non-profit Wikimedia Foundation.
Contributors can access the site can edit most of its articles.
Wikipedia is ranked among the ten most popular websites, and constitutes the Internet's largest and most popular general reference work.
A peer review of 42 science articles found in both Encyclopædia Britannica and Wikipedia was published in Nature in 2005, and found that Wikipedia's level of accuracy approached Encyclopedia Britannica's~\cite{Giles2005} \todo{rewrite}.

\todo{CITEME\@: Bill Tancer (May 1, 2007). ``Look Who's Using Wikipedia''. Time. Retrieved December 1, 2007. The sheer volume of content [\ldots] is partly responsible for the site's dominance as an online reference. When compared to the top 3,200 educational reference sites in the US, Wikipedia is No. 1, capturing 24.3\% of all visits to the category. Cf. Bill Tancer (Global Manager, Hitwise), ``Wikipedia, Search and School Homework'', Hitwise, March 1, 2007.}

In this section we describe the various entities that compose Wikipedia which are relevant to our analysis.

\paragraph{Pages}
The basic block of Wikipedia is the \emph{page}.
Pages are plain text document that can be customized using the wiki markup language, which is rendered when a user requests it.

A Wikipedia \emph{article}, or entry, is a page that has encyclopedic information on it.
A well-written encyclopedia article identifies a notable encyclopedic topic, summarizes that topic comprehensively, contains references to reliable sources, and links to other related topics.

\paragraph{Namespaces}
A Wikipedia \emph{namespace} is a set of Wikipedia pages whose names begin with a particular reserved word recognized by the MediaWiki software (followed by a colon).
For example, in the user namespace all titles begin with ``User:''.
In the case of the article (or main) namespace, in which encyclopedia articles appear, the reserved word and colon are absent.

At the time of writing, Wikipedia has 35 current namespaces: 16 subject namespaces, 16 corresponding talk namespaces, 2 virtual namespaces, and 1 special namespace.

Pages under the virtual namespaces (``Special'' and ``Media'') are not actually stored in the wikipedia database.
Special pages are created on demand by the MediaWiki software: for instance, ``Special:Log'' lists the log of all the activities on Wikipedia.
Files like images, videos and other assets live under the ``Media'' namespace.

\paragraph{Contributors}
Every time a \emph{contributor} edits a page, a new \emph{revision} is created.
It contains the text of the new version, as well as information about the contributor, the timestamp, the comment and others.

Contributor can either be humans or bots.
Bots are usually employed to to repetitive and mundane tasks, such as correcting obvious typos, updating listified pages of categories or revert vandalisms.

\subsection{Redirects}
A redirect is a page that has no content itself but sends the reader to another page, usually an article or section of an article.
Redirects can be created for many reasons, among which:
\begin{description*}
    \item[Alternative names] An entity may be known as different names.
    For instance, the page \emph{Edison Arantes do Nascimento} is a redirect to \emph{Pelé}, which is the most appropriate article title according to Wikipedia rules.
    \item[Abbreviations] Many terms have an abbreviate form, but the title should be the the extended name.
    For instance, \emph{US} redirects to \emph{United States}.
    \item[Alternative spellings] There are names that refer to the same entity, for example \emph{Color} and \emph{Colour}.
    \item[Likely misspellings] Some names are commonly misspelled, and in order to not give an error page to the user, they redirects to the correctly spelled page.
    \item[Technical reasons] Some redirects are created because of technical issues.
    For example, there are redirects for pages with non-ASCII letters that have only ASCII letters in the title.
    Another reason is because of the software underneath Wikipedia.
    First versions of ``UseModWIki'' needed pages to use the ``CamelCase'' convention for the name~\cite{wiki:camelcase}.
    This is not an issue anymore, but old page names are kept and they redirect to the non-camelcase name.
\end{description*}

To mark a page as a redirect, it is necessary to include the string \textbf{\#redirect: [[Target]]}, where \textbf{Target} is the page to which the reader will be redirected.

A recent study~\cite{Hill2014} shows that redirect pages play an important role in Wikipedia, and must keep in consideration when analyzing it.
We have replicated the results of the paper (see Table~\ref{tbl:redirects}) using the latest data available and results are consistent with the one given by the paper: the 59,25\% of Wikipedia articles are redirects.
This is particularly important in our analysis when we analyze the views of pages, because we must follow the redirects to get the correct count.
\todo{Maybe this needs an more detailed view in the results chapter?}

\subsection{Vandalism}
As a wiki, every one of its articles can be edited by anyone – there is no credential check ing.
Changes are visible to everyone immediately, without any review cycle.
This comes at a cost: \emph{vandalism}.
Vandalism may come in different forms: pages may be subject to malevolent edits, deletions or moves.
Priedhorsky et al.~\cite{Priedhorsky2007} studied this phenomenon and recommended policies that may help to mitigate the problem.
The paper shows some blatant examples of vandalism: for example, Adam Curry allegedly altered pages about the history of podcasting in order to promote his role and diminish that of others, Jeffrey Seigenthaler's article stated falsely for months that he was involved in the John F. Kennedy assassination, and the comedian Stephen Colbert has even conducted humorous tutorials on how to vandalize Wikipedia.

Wikipedia currently uses various techniques to reduce its impact.
Pages that are considered ``at risk'' are protected in a way that only some users can edit the content.
Furthermore, administrators can prevent contributors from editing pages by banning them.
This, however, does not resolve the problem because a contributor can change its IP address or create a new fake account.

This is a serious problem for any analysis and requires some extra effort to reduce the impact it has in our results.

\begin{table}[]
\centering
\caption{Count of Wikipedia redirects as of 1/9/2015}
\label{tbl:redirects}
\begin{tabular}{llll}
\multicolumn{1}{c}{\textbf{}}          & \multicolumn{1}{c}{\textbf{\# Redirects}} & \multicolumn{1}{c}{\textbf{\# Pages}} & \textbf{Percentage} \\ \hline
\multicolumn{1}{l|}{All namespaces}    & 8350978                                & 37178009                               & 22,46\%             \\
\multicolumn{1}{l|}{Article namespace} & 7075838                                & 11941857                               & 59,25\%
\end{tabular}
\end{table}



\section{Wikimedia page view statistics}
\label{sec:pagecounts}
Back in 2007, Domas Mituzas, a long-time volunteer database administrator for the Wikimedia Foundation, started generating statistics about the views of Wikimedia projects.
These statistics are now maintained by the Analytics Team\footnote{\url{https://www.mediawiki.org/wiki/Analytics}}.
The dataset contains the number of non-unique visitors for each page for all the Wikimedia projects, aggregated by hour.

Unfortunately this dataset needs a lot of cleaning and, in order to be used effectively, needs to be reordered, as described in Sect.~\ref{sec:reordering_pagecounts}.

\section{Microsoft Academic Service}
\label{sec:mag}
Microsoft Academic Search is an experimental research service developed by Microsoft Research to explore how scholars, scientists, students, and practitioners find academic content, researchers, institutions, and activities.
According to the authors, the service has already indexed 83 million papers achieving an above-95\% accuracy~\cite{Sinha2015}.
The service can also display the key relationships between and among subjects, content, and authors, highlighting the critical links that help define scientific research.

For these reasons, we use this service in our analysis.
The dataset is available under the name of ``Microsoft Academic Graph''\footnote{\url{http://research.microsoft.com/en-us/projects/mag/}}.

The available entities relevant to our research are the following:
\begin{description*}
    \item[Papers] Contains information like the name of the paper, the publication date, its \ac{DOI} (if available), the journal or the conference where it appeared on.
    \item[PaperReferences] Contains the references between papers.
    \item[Journals] Contains the list of journals.
    \item[Conference] Contains the list of conferences.
\end{description*}

\todo{Explain that the ``conferences'' dataset is extremely biased: it seems that there are only compsci conferences}

\todo{Should the problems of the datasets be explained here or in chapter 2?}

% \section{Big data}
% \label{sec:bigdata}
%
% \todo{Explain what big data is.}
%
% The datasets used in our analysis belongs to the ``big data'' category.
% In fact, some of them are so large that a single processor machine can not analyze them in a reasonable time.
% For instance, as described in Sect.~\ref{sec:reordering_pagecounts}, the cleaning and reordering of page views could take years \todo{verify this} to complete on a single machine.
% To solve this problem, we make use of the UniTN Cisca cluster processing power to parallelize the task.


\section{Related work}
\label{sec:relatedwork}
