% !TEX root = thesis.tex
\cleardoublepage{}
\chapter{Contributions}
\label{cha:Contributions}
In this chapter we describe the dataset generated and the software developed which are made available for further works.

\section{Page views ordered by project, article}
\label{sec:contrib_datasets_pagecounts}
This dataset~\cite{Bogon2016} has been generated starting from the Wikimedia's pagecounts-raw dataset using the program described in Section~\ref{sub:Sorting pagecounts}.
It contains the page view statistics for all the Wikimedia projects in the year 2014, ordered by $(project, page, timestamp)$.
We plan to generate this collection also for the remaining years.
The size is 4\,728 GB (428 GB compressed) split across 67056 files.

The content of the dataset is formatted in CSV, where each row describes the number of requests for a page of in a specific hour.
The fields are the following:
\begin{itemize}
    \item project: the name of the Wikimedia project, lowercase
    \item page: the name of the page, URL-escaped
    \item timestamp: the timestamp of the hour, (format \mintinline{text}{%Y%m%d-%H%M%S})
    \item count: the number of times the page has been requested
    \item bytes: the number of bytes transferred
\end{itemize}

We also released a Python library, \emph{pagecounts-search}\footnote{\url{https://github.com/youtux/pagecounts-search}}, to search through this source of information.
It can be used either as a library inside another project or directly as a command line tool.

This dataset has been created because we found that the original page views dataset is not well suited for most of the purposes.
Indeed some of the cited works use dodge the problem by restricting the number of pages to be analyzed or by using alternative sources.

\section{Wikipedia extraction framework}
\label{sub:contrib_programs_framework}
The \emph{wikidump}\footnote{\url{https://github.com/youtux/wikidump}} framework started as a project to extract some features from the Wikipedia dumps described in Section~\ref{sec:Wikipedia dumps}.
It uses libraries developed by Aaron Halfaker to analyze the XML structure and to extract identifiers from the text using regular expressions and parsers.
We have modified the identifiers extraction module to improve its performance, and it is currently waiting to be merged in the \emph{python-mwrefs}\footnote{\url{https://github.com/mediawiki-utilities/python-mwrefs}} library.

The framework allows the extraction of features such as the history of publication identifiers (\ac{ISBN}, \ac{DOI}, \ac{PMID}, \emph{arXiv}), statistics about the pages' sections, interlinks between pages, etc.

It analyzes the dumps by iterating over the elements of the XML file, without keeping them in memory in order to avoid exhaustion.
It also automatically decompresses the input file if needed and the user can choose to enable the compression of the output file.
Because of the nature of the dataset, the program can be used to analyze more files in parallel.
